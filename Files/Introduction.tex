\section{Introduction}
\label{sec:introduction}
\subsection{Background}
\subsubsection{Maintenance}
As industries globally strive for increased optimization and efficiency, the remarkable surge in computational power and technological advancements has made it possible to automate "higher-level" tasks that traditionally demanded human intellect. This progress significantly advances the development of unmanned autonomous vessels within the maritime sector, bringing them closer to mass production.

The practicality of autonomous vessels relies on maintaining continual awareness of machinery performance and operational status within the engine room. Examination of industry practices reveals that reliability in this sector has historically depended heavily on trial-and-error testing procedures. Most reliability research in the industry focuses on two distinct phases of a product's life cycle. Firstly the warranty period, during which most failures result from product malfunctions or quality-related issues. Secondly wear-out period, where failures primarily occur due to excessive wear and use \cite{thomas_warranty_1999}.

By utilizing sensors and logging software, equipment conditions are continually assessed, enabling efficient data analysis and the planning of predictive maintenance directly on vessels. The electric motor, as the primary device converting electrical energy into mechanical energy, finds extensive use in electric propulsion, powering thrusters for station keeping, and various onboard equipment across hundreds of ships. Notably, around 80-90\% of load installations rely on electric motors for their operation \cite{han_motor_2019}.

Forward-thinking organizations understand that maintenance should no longer be perceived solely as an expense; it must be seamlessly integrated into the business cycle. This integration ensures predictability, fosters growth, and enhances overall operational quality. Shifting from scheduled rule-based maintenance to on-condition maintenance and ultimately adopting a data-driven risk-based approach enables more precise and timely maintenance tasks. This evolved maintenance perspective offers numerous practical advantages, including cost reduction, improved safety, and increased availability of ship systems.

The practice of making failure predictions and determining the remaining useful life (RUL) yields substantial benefits, including work-style improvements, reduced crew workload due to autonomous monitoring, heightened safety through accident prevention, and the assurance of efficient and optimal operations. Future plans include the modular addition of monitoring devises for a holistic view of the whole engine room to continually enhance overall performance.

In the marine industry, traditional preventive and corrective maintenance approaches still prevail. Mechanical systems, including plants, machinery, and equipment components, are often replaced or overhauled at fixed intervals, even when they remain functional. This practice can lead to unnecessary replacement, repair, or maintenance expenses. Alternatively, these components may exceed their operational lifetimes before scheduled maintenance, resulting in system breakdowns and necessitating corrective maintenance. Consequently, these conventional maintenance methods are proving less effective in maintaining the reliability, safety, and maintainability of marine mechanical systems.

Predictive maintenance can be conducted even when no prior failure data for equipment is available. When such data is accessible, machine-learning models based on binary classification are employed to predict impending failures, facilitating timely repair or equipment replacement planning \cite{han_motor_2019}.

There are several parameters to be considered:
\begin{itemize}
	\item Temperature 
	\item Pressure
	\item Vibration
	\item Current
\end{itemize}

	
\subsection{Problem  statement}
 
Ship-related incidents often result in accidental errors, with a notable 62 percent attributed to human failure, specifically 22 percent linked to shipboard operations (Bratic et al., 2019). Engine room failures, a significant contributor to these incidents, can be traced to natural mechanical failures, electrical malfunctions, and human errors stemming from negligence or inadequate competency in engine room procedures. These failures garner heightened attention when they lead to severe consequences, such as oil spills causing environmental damage and the tragic loss of crew members' lives. Beyond the human and environmental toll, vessel owners also face substantial financial losses, including maintenance expenses and shipping costs, which could be allocated more efficiently.

While it is acknowledged that complete elimination of engine room accidents is unrealistic, this seeks to address the prevalent issues by introducing strategies that enhance efficiency and reduce downtime. Specifically, the proposal focuses on minimizing human errors through the automation of engine room motors and condition monitoring, aiming to achieve a level of autonomy that aligns with Level 5 standards, i.e(Hands-off, eyes-off, mind-off = human-off). This approach not only addresses the root cause of a significant portion of accidents but also aligns with the broader industry shift towards autonomous operations, ultimately promoting safety, environmental sustainability, and cost-effectiveness in maritime activities.\cite{noauthor_13_nodate}



\subsection{Objectives}
\subsubsection{Main Objective}
\paragraph{}  To monitor the health of ship motors thus improving reliability and preventing downtime
in ships. 
\subsubsection{Specific Objectives}
\begin{enumerate}
\item  To develop a condition monitoring framework for electric motors in ships.
\item Extraction of motor fault characteristics and signatures.
\item TO develop an algorithm that autonomously diagnoses motor faults.
%\item To model a modular framework onto which various equipment will be added to achieve predictive maintenance in the entire ship’s engine room.
%\item To design a product that has seamless integration on multiple motors.

\end{enumerate}
\subsection{Justification of the study}
Predictive maintenance offers a proactive approach to managing electric motors, crucial components in any facility. These motors are prone to various issues leading to disruptions, reduced productivity, and financial impact. By transitioning from manual inspections to automated methods, productivity increases, and errors decrease. Traditional maintenance practices, often involving premature replacements, can't entirely prevent unexpected downtime. The primary objective of predictive maintenance is to provide early alerts for potential motor issues, ensuring prolonged operational life. In the maritime context, motor failures lead to extensive downtime, impacting profitability. Advanced monitoring detects issues in advance, accurately determines repair times, and mitigates unexpected shutdown risks. Predictive maintenance, while complementing reactive and preventive approaches, represents a significant leap of up to 50\% forward in asset management, enhancing efficiency and minimizing disruptions for improved financial outcomes. \cite{sampaio_prediction_2019}.  

