\section{Introduction}
\label{sec:introduction}
\subsection{Background}
\subsubsection{Maintenance}
As the world’s industries push the boundaries of optimization and efficiency, the exponential increase in computational ability and technology, the automation of “higher-level” tasks that require human intellect is now possible. This headway brings unmanned autonomous vessels within the Maritime Industry closer to mass production. The practicality of autonomous vessels can only be achieved with constant awareness of the performance and operating state of machinery in the engine room (CBM). Observations of industry practices display that industry experience in reliability is heavily based on trial-and-error test procedures. Most of the reliability research in industry still focuses on two distinct periods of the product life. The warranty period, where most of the failures are due to product malfunctions or quality related problems, and, wear-out period, where the failures are due excessive wear and use \cite{thomas_warranty_1999}.

 Using sensors and logging software the condition of equipment is assessed as frequently as needed, enabling efficient analysis of data that facilitates planning of predictive maintenance on-board vessels. The electric motor is the most used device for conversion from electric to mechanical energy and is used for electric propulsion, powering thrusters for station keeping, and different on-board equipment on hundreds of ships. Typically, 80-90 percent of the load installations will be electric motors\cite{han_motor_2019}. Smart organizations know they can no longer afford to see maintenance as just an expense. Rather, maintenance must be integrated within the business cycle in order to guarantee predictability, growth and increase the overall quality of operations. Moving from a regime of scheduled rule-based maintenance via on-condition maintenance and ultimately to a data-driven risk-based regime can lead to more accurate and timely maintenance tasks. This smarter view of maintenance allows for achieving many practical advantages leading to lower costs and increased safety and availability of ship systems\cite{han_motor_2019}. 
 
 Making failure predictions and determination of remaining useful life (RUL), realizes significant benefits not limited to: work style reforms, reduction in crew workload in that monitoring is done autonomously, improved safety from preventing accidents before they happen, and ensuring efficient optimal operation. In future more equipment will be added in a modular manner to realize better optimal performance.  

With preventive and corrective maintenance still pronounced and used in the marine industry, mechanical systems such as plants, machinery and equipment components are being replaced or overhauled after some interval. Marine mechanical systems’ parts could be replaced within the fixed maintenance or scheduled interval when they are still functional thus leading to unnecessary replacement or repair or maintenance costs. Likewise, the Plants Machinery and equipment system parts may have exhausted their operation lifetime before the maintenance interval reaches. This may lead to the breakdown of the mechanical systems, thus resulting in corrective maintenance. Hence, such traditional maintenance approaches are becoming less effective towards the reliability, safety and maintainability of marine mechanical system.

Detection of operation anomalies is the kind of predictive maintenance that can be carried out even when no data from previous failures in the equipment is available. When available, machine-learning models based on binary classification are used to predict failures in the near future in order to plan repairs or substitution of equipment\cite{han_motor_2019}.

There are several parameters to be considered:
\begin{itemize}
	\item Temperature 
	\item Pressure
	\item Vibration
\end{itemize} 






The parameters temperature and pressure were considered because they are prime indicators of a motor’s operating performance. 

\subsection{Problem  statement}
 Out of 880 accidental errors in ship related incidents 62 percent are attributed to human failure; of this, 22 percent are shipboard related operations. Engine room failures are caused by a majority of three ways, either by: Natural mechanical failure, Electrical failure of components and whole systems, Human negligence, and poor competency in engine room procedures, Inaccurate diagnosis, and sub-par prevention measures. These accidents are more notably recognized when they result in internationally felt effects such as oil spills damaging large swathes of marine ecosystems or when loss of life of crew members is realized. But with greater significance but less spoken of - the loss of millions in profits in maintenance and shipping costs incurred to the vessel’s owner that would otherwise have been used more productively.
 
  While it is impractical to try and eliminate accidents in the engine room, this design proposal seeks to provide a solution to improving efficiency and mitigating downtime by implementing strategies to reduce human through the automation of engine room condition monitoring \cite{noauthor_13_nodate}. 


\subsection{Objectives}
\subsubsection{Main Objective}
\paragraph{}  To monitor the health of ship motors improving reliability and preventing downtime
in ships. 
\subsubsection{Specific Objectives}
\begin{enumerate}
\item  To develop a predictive maintenance algorithm for electric motors in ships.
\item To be able to determine the remaining useful life of electric motors. 
%\item To model a modular framework onto which various equipment will be added to achieve predictive maintenance in the entire ship’s engine room.
%\item To design a product that has seamless integration on multiple motors.

\end{enumerate}
\subsection{Justification of the study}
Electric motors serve as a critical component for any facility. However, electric motors can be prone to any number of issues that lead to motor faults and failures. Failures disrupt business operations, decrease productivity, and adversely impact a company’s bottom line. 

Motor inspection processes have shifted from manual scrutiny to semi-automated and fully-automated inspection. This will replace the time-consuming task of manual review, significantly increasing productivity while preventing missed inspection as well as errors. Traditionally maintenance involves routine inspection and repair done manually. This cannot completely prevent the risk of machine downtime and will also result in the unnecessary early replacement of usable parts \cite{sampaio_prediction_2019}.  


The purpose of this project is to alert about problems occurring in the motor and trying to mitigate the risk of unexpected failure. A well-planned predictive maintenance is the key to long life operation of motors. In ships unexpected failure causes downtime which deeply eats into profits. The traditional approach is to repair and replace equipment after a period of time but this cannot prevent downtime due to malfunctions, which put a halt to operations and incur massive losses. Advanced monitoring is implemented on parts that are about to break down and can be discovered in advance to accurately determine the time for repair and risk of unexpected shutdown can be prevented. \cite{sampaio_prediction_2019}

Although Reactive and preventive maintenance will always have a part in operations, Predictive maintenance is the next big step forward in the evolution of asset management. In fact, the ability to connect assets and feed information into a central system gives organizations the power to turn data into powerful insights and automatically take corrective, preventive or predictive action.
