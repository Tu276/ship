\section{Literature Review}
\subsection{Operation, Subsystems and Parameters}
Every day we rely on a wide range of machines, but every machine eventually breaks down unless it’s being maintained.  

The control, monitoring and maintenance of ships equipment's is fundamental for the quality and performance of productive process. Sensors and actuators play an important role in operation of various machines such as pumps, generators, among others, so they must always be in proper working condition. To guarantee this, machines are constantly monitored and types of maintenance of their components are performed. Corrective maintenance is performed in the case of a critical failure in the equipment and causes an unplanned downtime of the production line. Scheduled maintenance is performed periodically and equipment is checked and replaced, if necessary, in order to avoid unplanned downtime. Industry has begun to perform condition-based maintenance where predictive equipment status is used to plan a maintenance \cite{noauthor_predictive_nodate}.

Condition monitoring is a type of maintenance inspection where an operational asset is monitored and the data obtained is analysed to detect signs of degradation, diagnose the cause of faults, and predict for how long it can be safely or economically run. There are five general categories of Condition Monitoring techniques—vibration monitoring and analysis; visual inspection and non-destructive testing; performance monitoring and analysis; analysis of wear particles in lubricants and of contaminants in process fluids; and electrical plant testing. Condition Monitoring needs good quality data such as that obtained by carefully run tests. However, much useful information can often be obtained from a plant’s permanent instrumentation once repeatability is established\cite{lu_condition_2018}. The basic principle of condition monitoring is to indicate the occurrence of deterioration by taking physical measurements at regular intervals. Condition-based maintenance policy is used to inspect and replace the system according to the observed deterioration level. 

Predictive maintenance is referred to as the use of data, machine learning techniques, and statistical algorithms to predict the most likely failure outcome of systems \cite{tinga_predictive_2017}. The machinery data collected by monitors say sensors (wired or wireless), is analysed in order to provide a consistent forecast on when a given component or machinery should be replaced or maintained thus optimizing on maintenance cost and downtime. Predictive maintenance is a branch of condition-based maintenance. Condition based maintenance (CBM) is described as a maintenance practice that tolerates and detects the failure of a system during operation through continuous monitoring of the system during operation\cite{kimera_predictive_2020}. 

Marine vehicle managers and marine technical experts claim that installation of more sensors and cables for monitoring purposes, will only have an impact on maintenance the of marine systems only if the maintenance crew apply hand-in-hand other maintenance aspects like visual inspections, vibration measurements, regular oil analysis and enriching the maintenance information base with performance parameters \cite{kimera_predictive_2020}. 

Remaining Useful Life (RUL) is the time remaining for a component to perform its functional capabilities before failure. The concept of Remaining Useful Life (RUL) is used to predict life-span of components (of a service system) with the purpose of minimizing catastrophic failure events in both manufacturing and service sectors. This project involves acquiring real data from a normal working pump at its different stages in life. This will be done using multiple sensors attached to the pump at different times under different working conditions. Over time, the installed sensors will generate more and more data which can be used to improve the initial models and make near-perfect failure predictions. 

Currently the industry is majorly relying on sensors for condition monitoring which has facilitated decision making under time constraints. The time between the point where a potential failure occurs and the point where it deteriorates into a functional failure can be seen as an opportunity window during which decision-making algorithms can recommend actions with the aim to eliminate the anticipated functional failure or mitigate its effect. The system can record and monitor pressure and temperature conditions of an industrial motor and transmit the data through a wireless network to a data logging centre. The current prototype was developed using open-source software and hardware and can successfully identify abnormal motor conditions from sensor input values that exceed predefined set-points. 

The widespread use of sensors in industry allows for data acquisition, which, combined with advanced methods of analysis, can significantly improve and optimize production. Therefore, the data can be used not only to monitor the current state of the process and devices, but also to predict this state. The application of predictive methods to sensory data representing production process, including machine condition, allows for early identification and prediction of the faulty or hazardous process state or machine break down\cite{tinga_predictive_2017}.

The aim of this project is to present a data-driven method analysing sensory data to predict machine failure or identify its deteriorating condition, so that users can plan maintenance work and avoid unplanned downtime. This method was prepared based on real-life data with the assumption that the created solution will be used in industry as a result of the implemented project. 

Predictive maintenance is based on the prognostic and health management technology, which supposes that the remaining useful life of equipment can be predicted. However, due to uncertainty of prognostics there could be wrong decisions regarding the remaining time to failure. 

Currently, the most promising strategy of maintenance for various technical systems and production lines is the predictive maintenance (PM), which can be applied to any system if there is a deteriorating physical parameter like vibration, pressure, voltage, or current that can be measured\cite{sampaio_prediction_2019}.

Motors in various applications start deteriorating due to various reasons. Thus, the monitoring of their condition is of prime importance for sustaining the operation and maintaining efficiency. Motors are the backbone of industry; they start degrading due to different reasons such as long period of operation, variations of power supply, or harsh environment; which gradually lead to permanent damage. Consequently, it becomes crucial to monitor the operation continuously \cite{han_motor_2019}. 

Industry reliability surveys suggest that ac motor failures may be divided into five categories, including (IEEE, 1997): 

\begin{itemize}
	\item bearing: 44\% 
	
	\item stator winding: 26\% 
	
	\item rotor: 3\% 
	
	\item shaft: 5\% 
	
	\item others: 22\%. 
\end{itemize}


even though the replacement of defective bearings is the cheapest fix among the causes of failure, it is the most difficult one to detect. Motors that are in continuous use cannot be stopped for analysis. 



