\section{Literature Review}
\subsection{Operation, Subsystems and Parameters}
Every day we rely on a wide range of machines, but every machine eventually breaks down unless it’s being maintained.  

The control, monitoring and maintenance of ships equipment's is fundamental for the quality and performance of productive process. Sensors and actuators play an important role in operation of various machines such as pumps, generators, among others, so they must always be in proper working condition. To guarantee this, machines are constantly monitored and types of maintenance of their components are performed. Corrective maintenance is performed in the case of a critical failure in the equipment and causes an unplanned downtime of the production line. Scheduled maintenance is performed periodically and equipment is checked and replaced, if necessary, in order to avoid unplanned downtime. Industry has begun to perform condition-based maintenance where predictive equipment status is used to plan a maintenance.[6]  

Condition monitoring is a type of maintenance inspection where an operational asset is monitored and the data obtained is analysed to detect signs of degradation, diagnose the cause of faults, and predict for how long it can be safely or economically run. There are five general categories of Condition Monitoring techniques—vibration monitoring and analysis; visual inspection and non-destructive testing; performance monitoring and analysis; analysis of wear particles in lubricants and of contaminants in process fluids; and electrical plant testing. Condition Monitoring needs good quality data such as that obtained by carefully run tests. However, much useful information can often be obtained from a plant’s permanent instrumentation once repeatability is established. [?]. The basic principle of condition monitoring is to indicate the occurrence of deterioration by taking physical measurements at regular intervals. Condition-based maintenance policy is used to inspect and replace the system according to the observed deterioration level. 

Predictive maintenance is referred to as the use of data, machine learning techniques, and statistical algorithms to predict the most likely failure outcome of systems [9]. The machinery data collected by monitors say sensors (wired or wireless), is analysed in order to provide a consistent forecast on when a given component or machinery should be replaced or maintained thus optimizing on maintenance cost and downtime. Predictive maintenance is a branch of condition-based maintenance. Condition based maintenance (CBM) is described as a maintenance practice that tolerates and detects the failure of a system during operation through continuous monitoring of the system during operation [10]. 

Marine vehicle managers and marine technical experts claim that installation of more sensors and cables for monitoring purposes, will only have an impact on maintenance the of marine systems only if the maintenance crew apply hand-in-hand other maintenance aspects like visual inspections, vibration measurements, regular oil analysis and enriching the maintenance information base with performance parameters.[10] 

Remaining Useful Life (RUL) is the time remaining for a component to perform its functional capabilities before failure. The concept of Remaining Useful Life (RUL) is used to predict life-span of components (of a service system) with the purpose of minimizing catastrophic failure events in both manufacturing and service sectors. This project involves acquiring real data from a normal working pump at its different stages in life. This will be done using multiple sensors attached to the pump at different times under different working conditions. Over time, the installed sensors will generate more and more data which can be used to improve the initial models and make near-perfect failure predictions. 

Currently the industry is majorly relying on sensors for condition monitoring which has facilitated decision making under time constraints. The time between the point where a potential failure occurs and the point where it deteriorates into a functional failure can be seen as an opportunity window during which decision-making algorithms can recommend actions with the aim to eliminate the anticipated functional failure or mitigate its effect. The system can record and monitor pressure and temperature conditions of an industrial motor and transmit the data through a wireless network to a data logging centre. The current prototype was developed using open-source software and hardware and can successfully identify abnormal motor conditions from sensor input values that exceed predefined set-points. 

The widespread use of sensors in industry allows for data acquisition, which, combined with advanced methods of analysis, can significantly improve and optimize production. Therefore, the data can be used not only to monitor the current state of the process and devices, but also to predict this state. The application of predictive methods to sensory data representing production process, including machine condition, allows for early identification and prediction of the faulty or hazardous process state or machine break down [9] 

The aim of this project is to present a data-driven method analysing sensory data to predict machine failure or identify its deteriorating condition, so that users can plan maintenance work and avoid unplanned downtime. This method was prepared based on real-life data with the assumption that the created solution will be used in industry as a result of the implemented project. 

Predictive maintenance is based on the prognostic and health management technology, which supposes that the remaining useful life of equipment can be predicted. However, due to uncertainty of prognostics there could be wrong decisions regarding the remaining time to failure. 

Currently, the most promising strategy of maintenance for various technical systems and production lines is the predictive maintenance (PM), which can be applied to any system if there is a deteriorating physical parameter like vibration, pressure, voltage, or current that can be measured.[6] 

Motors in various applications start deteriorating due to various reasons. Thus, the monitoring of their condition is of prime importance for sustaining the operation and maintaining efficiency. Motors are the backbone of industry; they start degrading due to different reasons such as long period of operation, variations of power supply, or harsh environment; which gradually lead to permanent damage. Consequently, it becomes crucial to monitor the operation continuously.[5] 

Industry reliability surveys suggest that ac motor failures may be divided into five categories, including (IEEE, 1997): 

\begin{itemize}
	\item bearing: 44\% 
	
	\item stator winding: 26\% 
	
	\item rotor: 3\% 
	
	\item shaft: 5\% 
	
	\item others: 22\%. 
\end{itemize}


even though the replacement of defective bearings is the cheapest fix among the causes of failure, it is the most difficult one to detect. Motors that are in continuous use cannot be stopped for analysis. 



\subsubsection{Wind Tunnel}
A wind tunnel is a large tube with air moving inside. This movement of air is usually done by powerful fans. 

The first wind tunnel was built by Francis Wenham in 1871. However, it was the Wright Brothers who were the first to show the value of the wind tunnel in aerodynamic design with their 1902 wind tunnel.  The Wright Brothers’ wind tunnel was largely made of wood, with a glass window on the top to look down through and see the force balance, from which the
lift and drag forces could be read. The wind tunnel was powered by a fan driven off a natural gas fueled engine. Their tunnel was square of 16" by 16"(about 407mm by 407mm), and 6 foot long (about 1829mm), with a maximum test speed of 35 mph (about 56 km/h).
\begin{center}
	\begin{figure}[!h]
	\centering
	\includegraphics[width=0.75\linewidth]{Figures/Fig2}
	\caption{Diagram of a typical wind tunnel}
	\end{figure}
\end{center}

Later in the early 20th century in Europe, the main users of wind tunnels were Gustave Eiffel in France and Ludwig Prandtl in Germany. Prandtl built the first closed circuit wind tunnel in 1908. Closed circuit wind tunnels are characterized by the recirculation of the airflow with very minimal exchange with the exterior. Open circuit wind tunnels on the other hand, have an airflow that follows a straight path and flows to the contracted zone where the test section is located and then passes through a diffuser, a fan section and an exhaust.

By the 1940’s supersonic wind tunnels were in use. In 1972 a cryogenic wind tunnel was built at NASA Langley by injecting liquid nitrogen into the wind tunnel to cool the gas. This lowered the viscosity and increased the Reynolds number, and this tunnel had the capability to match Reynolds and Mach numbers simultaneously up to Mach 1.2
\cite{fernandes_design_nodate}.

Today the largest wind tunnel in the world is the National Full-Scale Aerodynamics Complex at NASA's Ames Research Center, which has a test section of cross-section 80 ft by 100 ft (24 m x 31 m). The types of instruments in common use in wind tunnels include boundary layer rakes, tufts, pitot tubes, pressure sensitive paint, smoke, and static pressure taps.
\begin{center}
\begin{figure}[!h]
	\centering
	\includegraphics{Figures/Fig3}
	\caption{NASA wind tunnels used to test new airplane designs}
\end{figure}
\end{center}

NASA uses wind tunnels to test scale models of aircraft and spacecraft. Wind tunnels help NASA to test ideas of making airplanes better and safer. They are also used to help engineers in designing spacecraft that will work in other planets such as mars - the wind tunnel can be used to simulate objects in an atmosphere that's thinner than ours e.g. an atmosphere that's exactly like the Martian atmosphere. NASA has wind tunnels of different types and sizes. Some are low-speed wind tunnels, others are hypersonic i.e. they are made to carry out tests at 4,000 mph (6437 kph).
\begin{center}
     %&\vspace*{-4.0cm}
    \begin{figure}[!h]
\centering
\includegraphics{Figures/Fig4}
\caption{NASA wind tunnels used to test the design of heavy-lift rocket}
\end{figure}
\end{center}
\subsection{Force Balances}
For wind tunnel applications, the wind axes is used as the reference frame. Where, X axis points to the accelerated air; Z axis points downward and; Y axis points to the right in the direction of the wind. In the reference frame above, lift is in the negative z-direction, drag in the negative x-direction and, side force in the negative y-direction.

Moment components on the x,y,z axes are rolling moment, pitching moment and yawing moment respectively. A three-component force balance can be considered to measure the lift, drag and pitch (angle of attack).

Force balances can be external or internal. In external force balances the test section lies outside of the wind tunnel test section, whereas in internal force balances the balance is inside the model itself connecting the model to the support structure.

Several different types of external force balances are available for wind tunnel use
\cite{morris_force_2010}:
\begin{enumerate}
\item Wire
\item Platform
\item Yoke
\item Pyramidal
\end{enumerate}
\begin{center}
	\begin{figure}[!h]
	\centering
	\includegraphics{Figures/Fig6}
	\caption{Typical configurations for external force balances}
	\end{figure}
\end{center}
In the wire balance, the model under testing is suspended by wires each connected to an extensometer (a sensor that produces an electrical output when submitted to a load and deforms). The shortcoming of the wire balance is the large tare drag caused by the wires which is difficult to quantify. They are also not robust nor versatile enough compared with the other alternatives. 

The platform balance is relatively easy to construct, assemble and instrument. However, for this balance, forces and torques are coupled and the balance resolving center does not coincide with the center of the tunnel.
 
In the yoke balance configuration, forces and torques are coupled and the balance resolving center coincides with the center of the tunnel. This configuration, however, presents some structural deflections due to the large span of the measuring and support arms.

The pyramidal balance configuration is a further improvement of the yoke balance in order to overcome the shortcomings of the other balances. It is capable of measuring six components of forces and
torques separately and without coupling,provided that the balance is well assembled and calibrated.

The different kinds of internal balances can be made based on:
\begin{enumerate}
\item The type of transducer i.e. strain gauge or piezoelectric balance.
\item Shape i.e. box balance and sting balance
\end{enumerate} 
\begin{center}
	\begin{figure}[!h]
	\centering
	\includegraphics{Figures/Fig7}
	\caption{Typical configurations for internal force balances. \textit{Left to right:} Box balance and sting balance.}
	\end{figure}
\end{center}
The box balance presents a cubic shape and can either be made of a solid piece of material or from assembled parts. In this configuation, the loads are transferred from the top to the bottom. The sting balance presents a cylindrical shape and the loads are transferred from one end to the other in the longitudinal direction. It can be used to measure forces or torques.

The advantage of internal force balances is that they minimize the interference caused by the supporting bars in the flow.
\subsection{Sensors}
\subsubsection{Load Sensors}
Several methods can be used to measure forces and torques in a force balance. These methods can be generally grouped into two:
\begin{enumerate}
\item Hydraulic measuring techniques.
\item Electric measurement techniques.
\end{enumerate}
Electric measurement techniques are preferred for Force balance applications. One such electric measurement device is the strain gauge. A strain gauge is an electromechanical device whose electrical resistance changes linearly with the strain in the component.

Metal foil strain gauges are widely used. This type of strain gauge provide more precise strain values than wire strain gauges. However, since the relative changes on electric resistance of the strain gauge are so small, it is necessary to develop an effective method to measure them because each strain gauge would require extremely accurate signal measurements. The solution is to have a set of strain gauges coupled in order to minimize the required accuracy, forming a force transducer i.e. the \textit{Wheatstone bridge}.
\begin{center}
		\begin{figure}[!h]
		\centering
		\includegraphics[width=0.6\linewidth]{Figures/Fig9}
		\caption{Wheatstone Bridge Circuit}
		\end{figure}
\end{center}
Load cells can also be used to measure the drag and lift forces.
\subsubsection{Attitude Sensor}
It is important to define the desired aerodynamic angles and to guarantee that they are measured accurately in relation to the air stream. One such angle is the angle of attack (\textalpha) shown in Figure 2.4. For this reason, specific devices that provide the attitude measurement should be implemented in order to improve the precision of the results.

Angle of attack (\textalpha)- angle measured between the longitudinal axis of the model and the direction of the flow on a vertical (Figure 2.4)
\begin{center}
	\begin{figure}[!h]
	\centering
	\includegraphics[width=0.6\linewidth]{Figures/Fig10}
	\caption{Angle of attack (\textalpha)}
	\end{figure}
\end{center}
\subsection{Summary of Gaps}
