\section{Introduction}
\subsection{Background}
\subsubsection{Stewart Platform}
\paragraph{} As the world’s industries push the boundaries of optimization and efficiency,
 the exponential increase in computational ability and technology The automation of “
 higher-level” tasks that require human intellect is now possible. This headway bring
 s unmanned autonomous vessels within the Maritime Industry closer to mass production
 . The practicality of autonomous vessels can only be achieved with constant awareness o
 f the performance and operating state of machinery in the engine room (CBM). Observa
 tions of industry practices display that industry experience in reliability is heavily based 
 on trial-and-error test procedures. Most of the reliability research in industry still
  focuses on two distinct periods of the product life. The warranty period, where most
   of the failures are due to product malfunctions or quality related problems, and, 
   wear-out period, where the failures are due excessive wear and use (?). Using 
   sensors and logging software the condition of equipment is assessed as frequently 
   as needed, enabling efficient analysis of data that facilitates planning of
    predictive maintenance on-board vessels. The electric motor is the most used device
	 for conversion from electric to mechanical energy and is used for electric 
	 
	 propulsion, powering thrusters for station keeping, and different on-board equipment on hundreds of ships. Typically, 80-90 percent of the load installations w
	 ill be electric motors.[1] Smart organizations know they can no longer afford to see
	  maintenance as just an expense. Rather, maintenance must be integrated within the 
	  business cycle in order to guarantee predictability, growth and increase the 
	  overall quality of operations. Moving from a regime of scheduled rule-based 
	  maintenance via on-condition maintenance and ultimately to a data-driven
	   risk-based regime can lead to more accurate and timely maintenance tasks. 
	   This smarter view of maintenance allows for achieving many practical advantages 
	   leading to lower costs and increased safety and availability of ship systems (2)
 This is shown in Fig.1.1.
\begin{center}
	\begin{figure}[!h]
	\centering
	\includegraphics{Figures/Fig1}
	\caption{General arrangement}
	\end{figure}
\end{center}
\paragraph{}Making failure predictions and determination of remaining useful life (RUL), realizes significant benefits not limited to: work style reforms, reduction in crew workload in that monitoring is done autonomously, improved safety from preventing accidents before they happen, and ensuring efficient optimal operation. [?] In future more equipment will be added in a modular manner to realize better optimal performance.  

With preventive and corrective maintenance still pronounced and used in the marine industry, mechanical systems such as plants, machinery and equipment components are being replaced or overhauled after some interval. Marine mechanical systems’ parts could be replaced within the fixed maintenance or scheduled interval when they are still functional thus leading to unnecessary replacement or repair or maintenance costs. Likewise, the Plants Machinery and equipment system parts may have exhausted their operation lifetime before the maintenance interval reaches. This may lead to the breakdown of the mechanical systems, thus resulting in corrective maintenance. Hence, such traditional maintenance approaches are becoming less effective towards the reliability, safety and maintainability of marine mechanical system 

Detection of operation anomalies is the kind of predictive maintenance that can be carried out even when no data from previous failures in the equipment is available. When available, machine-learning models based on binary classification are used to predict failures in the near future in order to plan repairs or substitution of equipment.[4]  These control means include:

There are several parameters tob considered: 
\begin{enumerate}
\item Use of two hydraulic jacks
\item Screw Jacks
\item Rotary actuator
\item Levers
\item Linear co-ordinate control
\item Strength
\end{enumerate}
\paragraph{}These shall be discussed in detail at a later stage.

\subparagraph{Application}
\paragraph{}The six DOF Stewart Platform provides an elegant design for simulating flight conditions which finds applications in the safe training of pilots. The mechanism differs from other simulators in that it has no fixed axes relative to the ground, and therefore within the limits of amplitude of the design it can truly simulate the conditions of banking by carrying the simulation of control surfaces into the axes of the new attitude.

\subsubsection{Wind Tunnel}
\paragraph{}
A wind tunnel is a large tube with air moving inside. This movement of air is usually done by powerful fans. The tunnel is used to copy the actions of an object in flight thus allowing
to obtain the components that better define this interaction, forces and moments.

\paragraph{} The first wind tunnel was built by
Francis Wenham in 1871. However, it was the Wright Brothers who were the first to show the value of the wind tunnel in aerodynamic design with their 1902 wind tunnel.  The Wright Brothers’ wind tunnel was largely made of wood, with a glass window on the top to look down through and see the force balance, from which the
lift and drag forces could be read. The wind tunnel was powered by a fan driven off a natural gas fueled engine. Their tunnel was square of 16" by 16"(about 407mm by 407mm), and 6 foot long (about 1829mm), with a maximum test speed of 35 mph (about 56 km/h).
\begin{center}
	\begin{figure}[!h]
	\centering
	\includegraphics{Figures/Fig2}
	\caption{Diagram of a typical wind tunnel}
	\end{figure}
\end{center}
\paragraph{}
Later in the early 20th century in Europe, the main users of wind tunnels were Gustave Eiffel in France and Ludwig Prandtl in Germany. Prandtl built the first closed circuit wind tunnel in 1908.
\paragraph{}Closed circuit wind tunnels are characterized by the recirculation of the airflow with with very minimal exchange with the exterior. Open circuit wind tunnels on the other hand, have an airflow that follows a straight path and flows to the contracted zone where the test section is located and then passes through a diffuser, a fan section
and an exhaust.
\cite{fernandes_design_nodate}
By the 1940’s supersonic wind tunnels were in use. In 1972 a cryogenic wind tunnel was built at NASA Langley by injecting liquid nitrogen into the wind tunnel to cool the gas. This lowered the viscosity and increased the Reynolds number, and this tunnel had the capability to match Reynolds and Mach numbers simultaneously up to Mach 1.2. 
\paragraph{}Today the largest wind tunnel in the world is the National Full-Scale Aerodynamics Complex at NASA's Ames Research Center, which has a test section of cross section 80 ft by 100 ft (24 m x 31 m). The types of instruments in common use in wind tunnels include boundary layer rakes, tufts, pitot tubes, pressure sensitive paint, smoke, and static pressure taps.
\begin{center}
\begin{figure}[!h]
	\centering
	\includegraphics{Figures/Fig3}
	\caption{NASA wind tunnels used to test new airplane designs}
\end{figure}
\end{center}

\paragraph{}NASA uses wind tunnels to test scale models of aircraft and spacecraft. Wind tunnels help NASA to test ideas of making airplanes better and safer. They are also used to help engineers in designing spacecraft that will work in other planets such as mars - the wind tunnel can be used to simulate objects in an atmosphere that's thinner than ours i.e. an atmosphere that's exactly like the Martian atmosphere. NASA has wind tunnels of different types and sizes. Some are low-speed wind tunnels, others are hypersonic i.e. they are made to carry out tests at 4,000 mph (6437 kph).
\begin{center}
     %&\vspace*{-4.0cm}
    \begin{figure}[!h]
\centering
\includegraphics{Figures/Fig4}
\caption{NASA wind tunnels used to test the design of heavy-lift rocket}
\end{figure}
\end{center}
\paragraph{}Wind tunnels are used to measure the aerodynamic forces on airplanes, wings, cars, trucks, bridges, and buildings, they can also be used to measure the
aerodynamic forces on sports balls, partially open valves, and anything else that can be mounted on the mounting sting. Wind tunnels are an effective tool used by engineers in determining the various aerodynamic loads due to movement of these vessels during the development process.
For wind tunnel testing with scale models to be applicable to the aerodynamics of the full-scale test object, conditions of dynamic similarity must be met.
\paragraph{}Wind tunnel testing is not cheap i.e. both to buid and to use. While a crude wind tunnel can be constructed relatively cheaply from a large fan and
sheet metal, our project will be limited to development of the six-degrees-of-freedom Stewart Platform and Force balance. We will use the low speed wind tunnel that is currently available at ?????
\subsubsection{Force Balances}
\paragraph{}A force balance is a device used to take direct measurement of forces and torques acting on the model that is being tested in the wind tunnel. The need for
force balances arises due to the necessity of having maximum load capability in all measuring components along with the accuracy for measuring minimum loads. Force balances can be external or internal. In external force balances the test section lies outside of the wind tunnel test section, whereas in internal force balances the balance is inside the model itself connecting the model to the support structure.
\cite{fernandes_design_nodate}
\paragraph{}Several different types of external force balances are available for wind tunnel use
\cite{morris_force_2010}:
\begin{enumerate}
\item Wire
\item Platform
\item Yoke
\item Pyramidal
\end{enumerate}
\begin{center}
	\begin{figure}[!h]
	\centering
	\includegraphics{Figures/Fig6}
	\caption{Typical configurations for external force balances}
	\end{figure}
\end{center}
\paragraph{}The different kinds of internal balances can be made based on:
1 the type of transducer i.e. strain gauge or piezoelectric balance.
2 shape i.e. box balance and sting balance
\begin{center}
	\begin{figure}[!h]
	\centering
	\includegraphics{Figures/Fig7}
	\caption{Typical configurations for external force balances}
	\end{figure}
\end{center}
\subsection{Problem statement}
\paragraph{}Simulation and analysis of scaled models is an important step in the development of aircrafts, vehicles and other machines . Such analysis provides aerodynamic peformance data that can be used to inform any modifications or improvements e.g. in aircrafts and vehicles to make them more efficient and safer. One such method that is used to perform aerodynamic performance evaluation is the wind tunnel, which can be low speed or high speed, used in conjuction with sensors for data aquisition by a computer. External or internal six-component force balances are also used. Another such technology that can be used for this purpose is the Stewart platform, which can be used to predict behaviour of vehicles and aircrafts in the actual environment.

\paragraph{}Whereas the wind tunnel gives very accurate results, it is expensive to build and use. Also, some objects require complex maneuver simulations to imitate the actual movements in air. There is therefore the need for dynamic positioning of objects in the wind tunnel.

\paragraph{}This project proposal, therefore, presents the development of a 3-component external force moment-balance to stand as a simple and economical alternative to the existing commercial solutions. The force balance should be able to measure lift, drag and pitching moment in small models and will be used with a generic low speed wind tunnel which is already available. The proposal also presents the design of a six-degrees-of-freedom Stewart platform to simulate the different movemnets of objects.
\subsection{Objectives}
\cite{stewart_platform_1965}
\subsection{Justification of the study}
\paragraph{}Additive manufacturing offers the ability to produce intricate products and parts with lower development costs, shorter lead times, less energy consumed during manufacturing as well as less material waste. This method can be used to manufacture delicate components such as the bipolar plates with elimination of the risks involved such as breakage of brittle Graphene material during production.     
\paragraph{}Precise control of reactant flow and pressure, stack temperature, and membrane humidity will increase the fuel cell’s robustness as well as efficiency.
\paragraph{}The goal of this research is to develop physic-based dynamic models of fuel cell systems and fuel processor systems and then apply multivariable control techniques to study their behavior. The analysis will give insight into the control design limitations and provide guidelines for the necessary controller structure and system re-design.

\subsection{Objectives}
\subsubsection{Main Objective}
\begin{enumerate}
\item To develop an external Stewart platform force balance for a low speed wind tunnel 
\end{enumerate}
\subsubsection{Specific Objectives}
\begin{enumerate}
\item To design and fabricate a six-degrees-of-freedom Stewart platform
\item To develop a force balance for the Stewart platform
\item To obtain forces and moments from a test model
\end{enumerate}
