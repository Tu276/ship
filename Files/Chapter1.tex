\section{Literature Review}
\subsection{Operation, Subsystems and Parameters}
Every day, our reliance on various machines is indispensable, but inevitably, these machines will face breakdowns unless they undergo proper maintenance.

Efficient engine room performance is crucial for the control, monitoring, and maintenance of ship equipment. Sensors and actuators play a pivotal role in the operation of machinery like pumps and generators. It is imperative that these components remain in optimal working condition. To ensure this, constant monitoring takes place, and various maintenance types are implemented. Corrective maintenance is carried out when there is a critical equipment failure, leading to unplanned downtime. Scheduled maintenance is performed regularly, involving thorough checks and necessary replacements to prevent unforeseen disruptions. The industry has embraced condition-based maintenance, utilizing predictive equipment status to plan maintenance activities \cite{noauthor_predictive_nodate}.

Condition monitoring, a form of maintenance inspection, involves continuously monitoring operational assets and analyzing the acquired data. This analysis aims to detect signs of degradation, diagnose faults, and predict the remaining safe or economical lifespan of the equipment. Five general categories of Condition Monitoring techniques include vibration monitoring and analysis, visual inspection and non-destructive testing, performance monitoring and analysis, analysis of wear particles in lubricants and contaminants in process fluids, and electrical plant testing. Effective condition monitoring relies on high-quality data, often obtained through carefully conducted tests. Valuable information can be gleaned from a plant's permanent instrumentation, particularly when repeatability is established \cite{lu_condition_2018}. The fundamental principle of condition monitoring is to identify deterioration by taking physical measurements at regular intervals. Condition-based monitoring is employed to inspect and replace systems based on observed deterioration levels. 

Predictive maintenance involves harnessing data, machine learning techniques, and statistical algorithms to anticipate potential system failures. Monitored machinery data collected through sensors, whether wired or wireless, undergoes analysis to consistently forecast when specific components or machinery should be replaced or maintained. This predictive approach optimizes maintenance costs and minimizes downtime, positioning itself as a subset of condition-based monitoring (CBM). CBM is a maintenance practice that tolerates and detects system failures during operation through continuous monitoring \cite{kimera_predictive_2020}.

Marine industry professionals emphasize that the impact of increased sensor and cable installations for monitoring marine systems is contingent on the synchronized application of various maintenance aspects. Visual inspections, vibration measurements, regular oil analysis, and supplementing the maintenance information base with performance parameters are crucial for maximizing the effectiveness of predictive maintenance strategies \cite{kimera_predictive_2020}.

Remaining Useful Life (RUL) signifies the time a component has before experiencing functional failure. Predicting the RUL is crucial for minimizing catastrophic failures in both manufacturing and service sectors. In this context, real data from a normally functioning pump is acquired at different life stages using multiple sensors under various working conditions. Over time, the accumulated sensor data enhances the accuracy of predictive models, enabling near-perfect failure predictions.

Presently, the industry heavily relies on sensors for condition monitoring, enabling timely decision-making. The time between potential failure and functional failure offers a window for decision-making algorithms to recommend actions, aiming to eliminate or mitigate the anticipated failure. For instance, industrial motors' pressure and temperature conditions can be recorded and monitored, with data transmitted wirelessly to a centralized logging center. The current prototype, developed using open-source software and hardware, successfully identifies abnormal motor conditions by analyzing sensor input values exceeding predefined set-points. This integration enhances the industry's ability to proactively address and prevent potential failures \cite{kimera_predictive_2020}.

The widespread integration of sensors in various industries enables data acquisition, which, when coupled with advanced analytical methods, holds the potential to significantly enhance and optimize production processes. This data not only serves to monitor the current state of processes and devices but also plays a pivotal role in predicting potential failure states. Predictive methods applied to sensory data representing the production process, including machine condition, facilitate early identification and prediction of faulty or hazardous states, as well as machine breakdowns.

This project's objective is to introduce a data-driven method for analyzing sensory data to predict machine failures or identify deteriorating conditions. The goal is to empower users to plan maintenance proactively and prevent unplanned downtime. The methodology is developed based on real-life data, with the anticipation that the implemented solution will find practical application in the industry.

Predictive maintenance relies on prognostic and health management technology, assuming that the remaining useful life of equipment can be predicted. However, the uncertainty inherent in prognostics introduces the possibility of incorrect decisions regarding the remaining time to failure.

Presently, the most promising maintenance strategy for various technical systems and production lines is predictive maintenance (PM), applicable to any system with a measurable deteriorating physical parameter such as vibration, pressure, voltage, or current \cite{sampaio_prediction_2019}.

Motors serve as the backbone of industry but are prone to degradation due to factors like extended operational periods, power supply variations, or harsh environmental conditions, leading to gradual permanent damage. Consequently, continuous monitoring of their condition is vital for sustaining operations and maintaining efficiency \cite{han_motor_2019}.

Reliability surveys in the industry, as identified by the Institute of Electrical and Electronics Engineers (IEEE) in 1997 \cite{Cheng:2018aa}, indicate that AC motor failures can be classified into five main categories:

\begin{itemize}
	\item \textbf{Bearing:} 44\%
	\item \textbf{Stator Winding:} 26\%
	\item \textbf{Rotor:} 3\%
	\item \textbf{Shaft:} 5\%
	\item \textbf{Others:} 22\%
\end{itemize}

Despite advancements in technology, these categories still provide valuable insights into the prevalent causes of AC motor failures. It is noteworthy that the replacement of defective bearings remains the most cost-effective solution among the identified failure causes. However, detecting bearing issues continues to be a challenging task, particularly in motors that operate continuously without interruption for analysis.

As of more recent developments, ongoing research and technological advancements in motor health monitoring have introduced innovative methods and tools for detecting and addressing these failure modes. These may include enhanced sensor technologies, data analytics, and condition monitoring systems, contributing to more effective and proactive motor maintenance practices in contemporary industrial settings.
\subsection{Temperature Sensors}

\subsubsection{Temperature Sensor Types}

A temperature sensor, commonly utilizing a thermocouple or a resistance temperature detector (RTD), serves the purpose of providing temperature measurements through an electrical signal. This device gauges the hotness or coolness of an object, and its types can be broadly categorized based on the mode of connection, namely contact and non-contact temperature sensors \cite{noauthor_temperature_2019}.

Contact sensors, exemplified by thermocouples and thermistors, establish direct contact with the object under measurement. Conversely, non-contact temperature sensors operate by measuring the thermal radiation emitted by the heat source. These sensors find applications in challenging environments such as nuclear power plants or thermal power plants.

For the specific requirements of this project, a contact-type temperature sensor was chosen due to its suitability for collecting data from the bearings.

\subsubsection{Thermistors: Temperature-Sensitive Resistors}

A thermistor, derived from the words "thermal" and "resistor," is a temperature-sensitive resistor commonly employed as a temperature sensor. Unlike standard resistors with minimized temperature coefficients, thermistors intentionally possess a high temperature coefficient. Most thermistors exhibit negative temperature coefficients (NTC), signifying a decrease in resistance with increasing temperature. There are also positive temperature coefficient thermistors, known as PTC thermistors \cite{noauthor_learn_nodate}.

The DS18B20, a compact temperature sensor with a built-in 12-bit analog-to-digital converter (ADC), is notable for its ease of integration with Arduino digital inputs. Operating on a one-wire bus, this sensor requires minimal additional components for communication \cite{noauthor_learn_nodate}. The choice of the DS18B20 aligns with the project's objectives and facilitates efficient temperature data acquisition from the monitored bearings.

\subsection{Current Sensors}

\subsubsection{Overview of Current Sensors}

A current sensor serves the purpose of detecting and converting current into an easily measurable output voltage. This output voltage is directly proportional to the current flowing through the measured path.

When electric current flows through a wire or a circuit, two phenomena are harnessed for the design of current sensors: voltage drop and the generation of a magnetic field around the current-carrying conductor. As a result, two primary types of current sensing methods emerge: direct and indirect.

\textbf{Direct Sensing:} This method involves the measurement of the voltage drop associated with the current passing through passive electrical components. It is grounded in Ohm's law, where the relationship between current, voltage, and resistance is utilized for accurate sensing.

\textbf{Indirect Sensing:} In this approach, the measurement focuses on the magnetic field surrounding a conductor through which current passes. It is based on Faraday's and Ampere's laws, leveraging the principles of electromagnetic induction.

\subsubsection{Pressure Transducers}

Pressure transducers, available in various shapes and sizes, typically feature a cylinder-shaped center housing the diaphragm and the pressure measurement chamber. One end incorporates a pressure port, commonly threaded, bolted, barbed fitted, or open, while the other end facilitates signal transmission. These transducers play a crucial role in converting pressure variations into electrical signals for monitoring and control purposes. The diverse design options allow for flexibility in applications, catering to different pressure ranges and environmental conditions.


\subsubsection{Vibration Accelerometer}

A vibration accelerometer, a crucial sensor in engineering, industrial, and scientific applications, is utilized for measuring and detecting vibrations or accelerations in various objects or systems. One prevalent type of vibration accelerometer leverages the piezoelectric effect, where certain materials generate an electric charge when subjected to mechanical stress. This type of accelerometer converts mechanical vibrations into electrical signals, facilitating measurement and analysis.

Vibrational analysis, applied for condition monitoring and prognostics of electric motors, provides valuable insights into motor health and performance. Changes in the vibration signature of an electric motor can indicate various faults, including bearing wear, misalignment, or rotor imbalance. For instance, bearing wear might manifest as increased vibration amplitude, while misalignment can result in irregular frequency patterns.

In a noteworthy study conducted by Chen and Wang (2017), advanced signal processing techniques were employed for motor prognosis using vibrational analysis. Their method, based on wavelet packet decomposition and envelope analysis, effectively extracted fault-related features from motor vibration signals. This approach demonstrated promise in detecting faults at an early stage, enhancing the potential for proactive maintenance.

Furthermore, the integration of machine learning algorithms with vibrational analysis has been explored to improve prognosis accuracy. Support vector machines and principal component analysis have been successfully employed to classify vibration signals into different fault categories. This integration not only aids in precise fault identification but also contributes to the prediction of impending motor failures.

In practical applications, continuous monitoring of vibrations with accelerometers allows for the establishment of baseline patterns and the identification of deviations that may indicate developing issues. This proactive approach enhances overall motor reliability and minimizes unplanned downtime, emphasizing the importance of vibrational analysis and advanced signal processing in predictive maintenance strategies.