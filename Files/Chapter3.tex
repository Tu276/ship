
\section{Methodology}


\subsection{System Modelling}

The predictive maintenance algorithm for motors system will be obtained from governing
equations from which a transfer function will be generated from the linearized model. The
transfer function will be used to generate a state space model for the system.
The observed sources of faults and their relative frequency. Such sources can be the core
components of the machine or its various sensors (such accelerometers and flow meters).
The process measurements through sensors. The number, type and location of sensors,
and their reliability and redundancies all will create both algorithm and comparative
model.
The sources of faults will translate to observed symptoms. Such cause-effect analysis will
require extensive processing of data from the available sensors.
Physical knowledge about the system dynamics.will result in mathematical modeling of
the system and its faults and from the insights of data. Understanding system dynamics
will involve detailed knowledge of relationships among various signals from the machinery
(such as input-output relationships among the actuators and sensors), the machine operating
range, and the nature of the measurements (for example, periodic, constant or stochastic).
The ultimate maintenance goal, such as fault recovery or development of a maintenance
schedule

\subsection{ Simulations}
From the generated models on matlab, simulations will be performed using the different
controllers and the responses and other metrics will be plotted out for further analysis.
Metrics such as rise time, settling time and stochastic response will be observed to determine the system performance.
%\begin{equation}
% R_{P}^B = R_{Z}(\gamma)*R_{Y}(\beta)*R_{X}(\alpha)
%\end{equation}
%In matrix form:
%\[ R_{P}^B =
% \begin{bmatrix}
% c\beta c\gamma & s\alpha s\beta c\gamma - c\alpha s\gamma %& c\alpha s\beta c\gamma + s\alpha s\gamma\\
% c\beta s\gamma & s\alpha s\beta s\gamma + c\alpha c\gamma %& c\alpha s\beta s\gamma - s\alpha c\gamma\\
% -s\beta & s\alpha c\beta & c\alpha c\beta  
% \end{bmatrix}

\subsection{ Sensors}
Sensors will be used to collect data from the system as it runs. These include:
\begin{enumerate}
\item Humidity sensor
\item Temperature sensor
\item Flow rate sensor
\item Pressure sensors
\item Voltage sensor
\item Current sensor

These sensors will be used by the controller to observe system performance and optimize for each parameter as well as the performance requirements.
\end{enumerate}
\subsubsection{PID Control of the Stewart Platform}
Proportional-integral-derivative (PID) control is used to control the Stewart platform. A dynamic model of the system is implemented in MatLab/Simulink.
\subsection{ Data Analysis}
The data collected from the simulations and sensors will be analysed using custom software created using jupyter notebooks. Graphs will be generated to compare the performance of each controller and evaluation of the selected controller.
\begin{itemize}
\item External force sensor
\item Stewart Platform as a force sensor
\end{itemize}
\subsubsection{External Force Sensor}
In this case it would require at least 3 orthogonally positioned load cells measuring each force component. Each load cell would be mechanically linked to the model such that forces experienced on each axis are measured by each load cell. 
\begin{center}
	\begin{figure}[!h]
		\centering
		\includegraphics{Figures/modBal}
		\caption{Diagram of a force balance \cite{post_force_2010}}
	\end{figure}
\end{center}
This configuration is however bulky by requiring an additional external system for force measurements in addition to the stewart platform for positioning the model. This is however complemented by the simplicity in calibration of the load cells and does not require a complex force transformation matrix and other issues with force amplification created by the use of an integrated system.
\subsubsection{Stewart Platform as a force sensor}
In this configuration the stewart platform legs are used as force sensors by attaching strain gauges in the legs of platform. Similar work has been done by \cite{ferreira2015design} without the use of actuators as is proposed in this project. Using the stewart platform as a force sensor requires the actuators to be locked with zero degrees of freedom.

Four strain gauges are required for each leg for a full wheatstone bridge configuration. 
\begin{center}
	\begin{figure}[!h]
		\centering
		\includegraphics{Figures/loadConf}
		\caption{Strain Gauge Configuration \cite{noauthor_measuring_nodate}}
	\end{figure}
\end{center}
In this case as shown in the figure the load cells are able to measure the axial strain on each leg. R1 and R3 are active strain gauges measuring the compressive Poisson effect (–νe). R2 and R4 are active strain gages measuring the tensile strain (+e). The output generated from the wheatstone bridge is then amplified and read to determine the strain on each leg.

\paragraph{Force transformation matrix} 
In such a case the forces experienced at the top of the platform are distributed between the 6 legs and as result, a force transformation matrix is required to resolve the forces apllied on each axis as measured by each load cell on each leg. 

If the platform is acted upon by an external wrench {$\vec{F}_e, \vec{M}_e$}, for static equilibrium of the body, the external wrench is statically balanced by the six leg forces of the stewart platform. Representing the unit vector $\hat{I}_i$ along the i-th leg with respect to B, the leg force is given  by $\hat{I}_if_i$. Considering the force equilibrium of the platform along  three mutually perpendicular directions in B(XYZ), the following force equations can be obtained:

$(F_e)_x = f_1I_{1x} + f_2I_{2x} + f_3I_{3x} + f_4I_{4x} + f_5I_{5x} + f_6I_{6x}$

$(F_e)_y = f_1I_{1y} + f_2I_{2y} + f_3I_{3y} + f_4I_{4y} + f_5I_{5y} + f_6I_{6y}$

$(F_e)_z = f_1I_{1z} + f_2I_{2z} + f_3I_{3z} + f_4I_{4z} + f_5I_{5z} + f_6I_{6z}$

where $(F_e)_x$, $(F_e)_y$ and $(F_e)_z$ are the external forces on the platform along three mutually perpendicular directions x, y and z of the frame B, respectively.

The moment due to the forces $\hat{I}_if_i$ about the origin of B is $(\vec{b}_i x \hat{I}_i)f_i$. Considering the moment equilibrium about x, y and z axes of B, the following moment equations can be obtained:

$(M_e)_x = f_1(\vec{b}_1 x \hat{I}_1)_x + f_2(\vec{b}_2 x \hat{I}_2)_x + f_3(\vec{b}_3 x \hat{I}_3)_x + f_4(\vec{b}_4 x \hat{I}_4)_x + f_5(\vec{b}_5 x \hat{I}_5)_x + f_6(\vec{b}_6 x \hat{I}_6)_x$

}$(M_e)_y = f_1(\vec{b}_1 x \hat{I}_1)_y + f_2(\vec{b}_2 x \hat{I}_2)_y + f_3(\vec{b}_3 x \hat{I}_3)_y + f_4(\vec{b}_4 x \hat{I}_4)_y + f_5(\vec{b}_5 x \hat{I}_5)_y + f_6(\vec{b}_6 x \hat{I}_6)_y$

$(M_e)_z = f_1(\vec{b}_1 x \hat{I}_1)_z + f_2(\vec{b}_2 x \hat{I}_2)_z + f_3(\vec{b}_3 x \hat{I}_3)_z + f_4(\vec{b}_4 x \hat{I}_4)_z + f_5(\vec{b}_5 x \hat{I}_5)_z + f_6(\vec{b}_6 x \hat{I}_6)_z$

where $(M_e)_x$, $(M_e)_y$ and $(M_e)_z$ are the external moments on the platform  about the three coordinate axes of B. Combining the equations the relationship between the external wrench and the forces experienced by the legs can be expressed as follows:
$$
\begin{Bmatrix}
\vec{F}_e \\
\vec{M}_e \\
\end{Bmatrix} = [H]\{F\}
$$

\subsection{Velocity Measurement}
An important part in wind tunnel measurements is the measure of pressure at specific points in the wind tunnel and computing the correspondingair speed. This is achieved by the use of a pitot probe. 
\begin{center}
\begin{figure}
\centering
\includegraphics{Figures/pitot}
\caption[Pitot-static tube]{Pitot-static tube \cite{noauthor_wind_nodate}}
\end{figure}
\end{center}

The equation relates the speed of the fluid at a point to both the mass density of the fluid and the pressures at the same point in the flow field. For steady flow of an incompressible fluid for which viscosity can be neglected, the fundamental equation has the form:

$$ v = \sqrt{\frac{2(P_0 - P)}{\rho}}$$

Where V is the speed of the fluid, P0 is the total, also called the stagnation, pressure at that point of measurement, and p is the static pressure at the same point.

Three pitot probes are to be used in the wind tunnel, these are in the test section, intake and dissuser sections.